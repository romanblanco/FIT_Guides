\documentclass[a5paper,10pt]{article}
\usepackage[T1]{fontenc}
\usepackage[utf8]{inputenc}
\usepackage[top=0.5cm, left=0.5cm, text={13.5cm, 19cm}, ignorefoot]{geometry}
\usepackage{indentfirst, times}
\usepackage[czech]{babel}
\begin{document}

\section{Fulltexty}

\paragraph{Vysvětlete pojem "vkládání čekacích stavů do komunikace mezi zařízeními v systémové sběrnici". Proč je nutné touto technikou systémovou sběrnici vybavit? Jak je tento problém řešen ve sběrnici PCI?\hfill(8~bodů)}
\begin{itemize}
	\item řadič PZ musí být schopen dodat data na datovou část sběrnice v určitém čase
	\item pokud toho není schopný je nutné vkládat čekací stavy
	\item PCI to řeší pomocí stavů \texttt{IRDY} a \texttt{TRDY}
\end{itemize}

\paragraph{Popsat Zoned Bit Recording, čeho se dosáhne jeho použitím}
\begin{itemize}
	\item na stopách jsou rozdílné počty sektorů
	\item vnější stopy - více sektorů
	\item vnitřní stopy - méně sektorů
	\item plocha disku je rozdělena na zóny, v zóně je stejný počet sektorů, odlišný od počtu sektorů v jiné zóně, lepší využití vnějších stop
\end{itemize}

\paragraph{Vysvětlete princip kódování v rozhraní SATA. Jaké cíle jsou tímto kódováním sledovány?\hfill(8~bodů)}
\begin{itemize}
	\item využívá se kódování 8b/10b, snaha dosáhnout nulové stejnosměrné složky a požadované hodnoty RLL přenášeného signálu
	\item stejnosměrná složka umožňuje uchování disparity (počet 0 a 1) na vodiči
	\item pro využití detekce chyb na přijímací straně
	\item původních 256 8bitových znaků zakódováno do 10bitů
	\item 8bitů nabízí vytvořit 256 různých kombinací 10b až 1024
	\item z této množiny je pak možno vybrat potřebný počet kombinací, aby splňoval konkrétní požadavky
\end{itemize}

\paragraph{Vysvětlit pojmy: sdílená/nesdílená a dedikovaná/nededikovaná sběrnice, u sdílené a dedikované uvést příklady}
\begin{itemize}
	\item \textbf{Dedikovaná sběrnice} -- má pouze jeden účel a věnuje se pouze jedinému koncovému zařízení -- AGP
	\item \textbf{Nededikovaná sběrnice} -- více možných koncových zařízení
	\item \textbf{Sdílená sběrnice} -- příkaz, data a adresa se přenášejí po společné sadě vodičů. Musí zde být identifikační signály, kterými se rozliší jaký typ informace je v aktuálním okamžiku na sběrnici -- PCI
	\item \textbf{Nesdílená sběrnice} -- samostatné vodiče pro příkazy, data, adresu, každý odděleně -- ISA
\end{itemize}

\paragraph{Princip zjišťování stavu PZ - stavová slabika, slabika závad. Použití příkazu "zjisti závady".}
\begin{itemize}
	\item po ukončení PO musí řadič PZ zjistit jak PO dopadla
	\item nejdříve se přenese stavová slabika, ve které je bit "vznikla chyba"
	\item pokud je tento bit nastavený, musí se analyzovat jaká chyba nastala, dochází k přenosu "slabiky závad"
	\item řadič PZ se na slabiku dotazuje příkazem "zjisti závady", každý bit ve slabice závad odpovídá jednomu typu chyby
\end{itemize}

\paragraph{Třírozměrná adresace, LBA}
\begin{itemize}
	\item součást řadiče jsou registry válce, hlavy a sektorů
	\item pokud se pracuje s těmito adresami jedná se o třírozměrnou adresaci
	\item režim LBA, všechny tyto registry jsou viděny logikou řadiče jako jeden registr, v němž je umístěná adresa sektoru, tj. jednorozměrná adresace
	\item v řadiči je k dispozici indikace (programovatelná ze strany procesoru), že adresa je typu LBA
	\item řešení problému přístupu k větším kapacitám než umožňovala třítozměrná adresace
\end{itemize}

\paragraph{Rozdíl mezi programovatelným přerušením a přerušením u V/V operaci}
\begin{enumerate}
	\item PZ vyvolá požadavek na přerušení
	\item V/V rozhranní vyšle signál IRQ na řadič přerušení
	\item Řadič vygeneruje signál INTR "někdo žádá o přerušení" a odesílá na CPU
	\item CPU se zeptá (pomocí signálu INTA) o jaké zařízení se jedná
	\item[---] odtud je shodný postup s programovým přerušením ---
	\item CPU uloží stavové operace prováděného programu do zásobníku
	\item Ve vektoru přerušení nalezne adresu podprogramu a vykoná jej
	\item Obnoví se stav ze zásobníku
\end{enumerate}

\paragraph{S čím souvisí "Vybavovací doba" u diskových pamětí?}
\begin{itemize}
	\item vybavovací doba = režie provedení příkazu + doba vystavení + doba uklidnění + zpoždění vlivem otáček
\end{itemize}

\paragraph{"Čekací doba" diskové paměti souvisí?}
\begin{itemize}
	\item s rychlostí otáčení
\end{itemize}

\paragraph{Vyjmenujte komponenty, které se podílejí na realizaci periferní operace?}
\begin{itemize}
	\item počítač (CPU nebo řadič DMA), systémová sběrnice, adaptér, V/V sběrnice, periferní zařízení
\end{itemize}

\paragraph{Pro paralelní rozhraní platí, že s narůstajícím synchronizačním kmitočtem (rychlostí přenosu) se zvyšuje nebezpečí přeslechů mezi signály. Jak je tento problém řešen v dalších generacích sběrnic a rozhraní?}
\begin{itemize}
	\item přechodem na sériové rozhraní a diferenciálním signálem přes dva vodiče
\end{itemize}

\paragraph{Vysvětlete pojmy: fullduplex a halfduplex}
\begin{itemize}
	\item \textbf{Fullduplex:} možnost současně přenášet data v obou směrech s maximální možnou frekvencí
	\item \textbf{Halfduplex:} možnost přenášet pouze v jednom směru z
\end{itemize}

\paragraph{Vysvětlete princip fázového závěsu v technikách, kdy je synchronizace vložena do přenášených dat}
\begin{itemize}
	\item PLL, skládá se z fázového detektoru a oscilátoru
	\item Princip: porovnává fázi vstupního signálu s fází signálu produkovaného oscilátorem a upravuje frekvenci tak, aby se obě fáze shodovaly
	\item Použití: v záznamu metodou MFM
\end{itemize}

\paragraph{Princip fungování a struktura LCD. Jak řeší DVI přechod na vyšší rozlišení.}
\begin{itemize}
	\item Struktura:
	\begin{itemize}
		\item dvě skleněné vrstvy, mezi kterými jsou rozmístěny kapalné krystaly
		\item zadní sklo (blíže ke zdroji světla)
		\item přední sklo tvoří RGB filtry, tvoří barvu a její intenzitu podle toho jak jednotlivé složky barvy projdou kapalným krystalem
	\end{itemize}
	\item Princip: přechod světla je řízen kapalným krystalem
	\item DVI: další kanál nebo zavedení techniky "reduced blanking" = zmenšení nezobrazované oblasti nebo zmenšení zatemněné oblasti
\end{itemize}

\paragraph{Ve kterých zařízeních se vyskytuje problém clock-skew?}
\begin{itemize}
	\item PCI a DVI
\end{itemize}

\paragraph{Vysvětlete pojem "registry mapované do adresového prostoru operační paměti". Zobrazte dekodér, popište jeho vstupy/výstupy, vysvětlete jejich funkci. \hfill (9~bodů)}
\begin{itemize}
	\item Paměť i PZ mají stejný adresný prostor, lze používat širokou škálu instrukcí, omezenější počet adres
	\item je možné k registrům přistupovat stejně a stejnými instrukcemi jako k operační paměti
	\item Bity A3-A9 jsou přivedeny na komparátor, který v případě příchodu vyšších adres povolí dekodér rozvádějící povolení v závislosti na bitech A0-A2 na jednotlivé registry
	\item výhodou je velké množství instrukcí, nevýhodou menší vyhrazený adresní prostor -- omezené množství zařízení a registrů, nebo naopak omezený adresní prostor pro operační paměť
\end{itemize}

\paragraph{Synchronní sběrnice. Podle čeho posuzujeme, zda je sběrnice synchronní?}
\begin{itemize}
	\item operace jsou realizovány od jedné z hran synchronizačního pulzu: v okamžiku výskytu hrany se vyhodnocuje, zda se požadované operace může provést (připravenost obou zařízení)
	\item interval mezi dvěma nástupnými sousedícími hranami: hodinový cyklus nebo sběrnicový cyklus
	\item stab na sběrnici vyhodnocuji všechna zařízení (např. adresu vystavení na sběrnici)
	\item činnost je rozdílná pro zápisný cyklus nebo čtecí cyklus
\end{itemize}

\paragraph{Vysvětlete následující pojmy: doba vystavení, vybavovací, rotační zpoždění\hfill(8~bodů)}
\begin{itemize}
	\item \textbf{Doba vystavení:} čas, za který se hlavy dostanou na požadovaná data
	\item \textbf{Doba vybavení:} čas, za který je disk schopen předat data (doba vystavení + rotační zpoždění)
	\item \textbf{Rotační zpoždění:} čas, za který je provedena jedna půlotáčka plotny
\end{itemize}

\paragraph{Vysvetlit pojem izolovane vstupy/vystupy. Zobrazit dekoder, popsat jeho vstupy/vystupy, vysvetlit jejich funkci\hfill(9~bodů)}
\begin{itemize}
	\item registry mají vlastní adresný prostor
	\item existují dva disjunktní adresné prostory
	\item jsou k dispozici dvě skupiny instrukcí (IN/OUT)	
	\item v tomto adresním prostoru např. bity A3-A9 identifikují řadič PZ a zbývající bity A0-A2 registr v rámci tohoto řadiče
	\item obvodová realizace provedena komparátorem napojeným na adresové bity sběrnice A3-A9, který v případě shody povolí průchod zbývajících bitů k dekodérům a datové bity sběrnoce pustí ke vstupům všechnsvých registrů
	\item nevýhodou je omezená instrukční sada
\end{itemize}

\paragraph{Rozdíl mezi SATA a DVI}
\begin{itemize}
	\item Kódování:
	\begin{itemize}
		\item \textbf{SATA:} 8b/10b, vyběr pouze kombinací splňujících RLL, nulová stejnosměrná složka + počítání disparity
		\item \textbf{DVI:} jiný způsob 8b/10b, 9. bit signalizuje použití inverze pro minimalizaci počtu přechodů, 10. bit signalizuje provedení inverze bitu pro dosažení nulové stejnosměrné složky signálu
	\end{itemize}	
	\item Synchronizační rozdíly:
	\begin{itemize}
		\item \textbf{SATA:} Embedded clock, synchronizaci lze odvodit z dat
		\item \textbf{DVI:} Synchronizace vedena zvlášť (clock skew), jako součást rozhraní	
	\end{itemize}
\end{itemize}

\paragraph{O rozhraní IDE platí:}
\begin{itemize}
	\item signály rozhraní IDE jsou podmnožinou signálů systémové sběrnice ISA
	\item přes rozhraní IDE	jsou programově přístupné registry řadiče PZ (např. pevného disku)
\end{itemize}

\paragraph{Pojem vertikální zpětný běh:}
\begin{itemize}
	\item návrat paprsku zpět na začátek obrazovky
	\item počet zpětných běhů za sekundu = počet zobrazených snímků za sekundu	
\end{itemize}

\paragraph{Nejhorší vzorek dat:}
\begin{itemize}
	\item vzorek s největším počtem změn magnetizace
	\item FM -- samé jedničky
	\item MFM -- samé jedničky nebo nuly 	
\end{itemize}

\paragraph{Adresace na SCSI a jak se dle toho řeší distributivní přidělování sběrnice, který prvek má ve SCSI nejvyšší prioritu, kolik je možné přidat zařízení do 16b sběrnice SCSI, struktura SCSI.}
\begin{itemize}
	\item adresa je 8b, z toho pouze jeden bit je 1, určuje které zařízení má prioritu obsluhy
	\item omezený počet adres
	\item lze připojit pouze 7 zařízení
	\item distribuovaná přidělovací sběrnice (bez arbirta, menší priorita ruší žádost)
	\item ID7 (nejvyšší adresa) host adapter
	\item do 16b sběrnice je možné přidat 15 zařízení, 16té je adapter
\end{itemize}

\paragraph{Vysvětlit princip komunikace dotaz-odpověď u paralelní a sériové sběrnici.}
\begin{itemize}
	\item Na úrovni signálů paralelní sběrnice:
	\item Na úrovni packetu v seriových sběrnicích:	
\end{itemize}

\paragraph{Přímý přístup do paměti (DMA)}
\begin{itemize}
	\item data se přenášejí z řadiče PZ do operační paměti přes sběrnice bez procesoru
	\item data se nepřevádějí přes řadič DMA, řadič jen řídí přenos
	\item využíván pro přenos mezi disketovou a operační pamětí	
\end{itemize}

\paragraph{PCI express}
\begin{itemize}
	\item Vysokorychlostní, seriová, nízko napěťová sběrnice, dvojitý jednosměrný přenos - full duplex
	\item 8b/10b kódování, point-to-point, síť spojů	
\end{itemize}

\paragraph{AGP}
\begin{itemize}
	\item rozšíření PCI, je rychlejší, dedikovaná, využívá ji grafický adaptér
	\item zřetězení adresování: adresy jsou zasílány zřetězeně, poté nastává souvislé čtení dat	
\end{itemize}

\paragraph{CRT monitor}
\begin{itemize}
	\item využívá se elektronkové dělo
	\item černobílé zobrazení:
	\begin{itemize}
		\item odstín barvy dle intenzity energie
		\item černá/bílá = 0/1
		\item stupně šedi = počet hodnot závisí na počtu reprezentačních bitů o barvě
	\end{itemize}	
	\item barevné zobrazení:
	\begin{itemize}
		\item analogová reprezentace
		\item aditivní míchání barev, jednotlivé body RGB leží vedle sebe, lidské oko je nedokáže rozlišit, vnímá celek jako barvu	
	\end{itemize}
	\item digitální zobrazení: barva RGB se buď podílí nebo nepodílí na výsledné barvě
	\item analogové zobrazení: každá složka má svou úroveň jasu, využití DA převodníku pro převod informace o barvě na analogovou VGA -- 1 barva 6b, SVGA -- 1 barva 8b
\end{itemize}

\paragraph{Clock-skew}
\begin{itemize}
	\item zpoždění datových signálů oproti synchronizaci
	\item nastává při zvyšování kmitočtu
	\item řešením je zrušit přenos synchronizace jako samostatného signálu -- odvození dat z vnitřní synchronizace	
\end{itemize}

\paragraph{TMDS}
\begin{itemize}
	\item snaží se omezit počet přechodů mezi 1 a 0
	\item používá se při přenosu před DVI
	\item diferenciální signál -- vyšší odolnost proti rušení
	\item každá barva na jeden dvoudrátový spoj -- dva kanály k dispozici	
\end{itemize} 
 
\paragraph{Cylinder skew}
\begin{itemize}
	\item posunutí číslování sektoru mezi cylindry
	\item čteme od středu disku, po skončení čtení, musíme přestavit na další cylinder
	\item sektory jsou mezi sebou posunuty
	\item je dostatek času na vystavení
	\item v relaci musí být rychlost otáčení a vystavení
	\item pokud je rychlost otáčení velká, musí být posunutí mezi sektory větší
\end{itemize}

\paragraph{Při DMA režimu je na adresovou sběrnici vystavena:}
\begin{itemize}
	\item Adresa paměti	
\end{itemize}
 
 
\paragraph{Vysvětlit "přerušování spouštěné hranou". V jaké sběrnici se to vyskytuje. A něco o tom, zda je možné tyto přerušení sdružovat do stejného vodiče.}
\begin{itemize}
	\item detekování žádosti o přerušení a její následná obsluha jsou spouštěny hranou žádosti o přerušená
	\item generování se odehraje na základě přechodu z "provádění PO" do stavu "PO skončila"
	\item každá žádost o přerušení je vedena do počítače přes svůj vlastní a nesdílený kontakt v konektoru systémové sběrnice = není možné sdružovat do jednoho vodiče více žádostí
	\item využita na sběrnici ISA		
\end{itemize}

\paragraph{Přerušení spouštěné úrovní, jaká sběrnice, je možné je sdružovat do stejného vodiče?}
\begin{itemize}
	\item více zařízení může žádat o přerušení ve stejnou dobu, procesor zjišťuje, kdo o co žádal, protože má omezený počet vodičů pro "žádost o přerušení"
	\item generování žádosti -- signál je na úrovni 0
	\item na všech zařízeních je signál realizován jako výstup s otevřeným kolektorem
	\item sběrnice PCI	
\end{itemize}

\paragraph{Popište architekturu "počítač s V/V procesorem". Kde je uplatněn a jak -- zobrazte obrázkem.\hfill(8~bodů)}
\begin{itemize}
	\item procesor nemá přístup k registrům řadiče -- nemůže do nich zapisovat ani jejich obsah číst
	\item V/V procesor komunikuje s řadiči jednotným způsobem formou tzv signálních sledů
\end{itemize}

\paragraph{Ze kterých sekcí sestává systémová sběrnice, popište funkci těchto sekcí. Popište princip "přenos slova dat z universálního registru procesoru do registru jiného klienta sběrnice" realizované na základě instrukce. Jak jsou tyto registry adresovány (přes adresovou část systémové sběrnice či jinak)?\hfill(9~bodů)}
\begin{itemize}
	\item CPU, paměť, DMA řadič, HDD řadič \hfill \textbf{asi} 
\end{itemize}


\paragraph{Vysvetlit duvody pro zavedeni radice preruseni do architektury pocitace. Vypsat zakladni principy obsluhy zadosti o preruseni vcetne role systemove sbernice\hfill(8~bodů)}
\begin{itemize}
	\item Pokud periferní zařízení potřebuje vyhodnotit stav, který na něm nastal, je potřeba to nějak sdělit CPU
	\item CPU, které v zpracovává určité požadavky programu, nemůže kontrolovat ještě stav na periferních zařízeních
	\item proto je zaveden řadič přerušení, který kontroluje stavy PZ a pokud nějaké žádá o přerušení, uvědomí o tom CPU, které náležitě zareaguje
	\item 2 způsoby přerušení:
	\begin{itemize}
		\item[1.] \emph{Přerušení spouštěná hranou} -- detekován přechod "Provádění PO" do "PO skončila", signál beží po jednom vodiči pro každé PZ 
		\item[2.] \emph{Přerušení spouštěná úrovní} -- na základě obdržení "žádosti o přerušení" se zjišťuje, kdo žádal, protože o přerušení může žádat ve stejnou chvíli více zařízení
	\end{itemize}	
\end{itemize}

\paragraph{Popsat slabikovy a blokovy prenos dat. Popsat jak vypada struktura prenasene informace u obou principu. uvest priklad sbernice, kde se pracuje s blokovym prenosem dat. Kterym z techto principu se dosahuje vyssich objemu prenasenych dat?\hfill(8~bodů)}
\begin{itemize}
	\item \textbf{Blokový přenos:} přenášejí se souvislé bloky dat
	\begin{itemize}
		\item sdílená sběrnice, adresa se posílá jednou a data následují za sebou
	\end{itemize}
	\item \textbf{Slabikový přenos:} přenos po slabikách	
\end{itemize}

\paragraph{Pojmy z usb vysvětlit: device, function}
\begin{itemize}
	\item \textbf{Device} -- zařízení, které neposkytuje konkrétní služby, ale pouze rozšiřuje služby sběrnice (např. HUB)
	\item \textbf{Function} -- zařízení, které poskytuje nějaké konkrétní služby (např. myš, klávesnice, tiskárna)	
\end{itemize}

\paragraph{Jak zrychluje přenos nízká úroveň napětí signálu, jak zvyšuje spolehlivost použití diferenciálního spoje}
\begin{itemize}
	\item Nízká úroveň napětí umožňuje rychlé přepínání mezi stavy v TTL, technika LVDS -- Low Voltage Differential Signaling
	\item Diferenciální spoj posílá signál a jeho invertovanou podobu dvěma vodiči, na přijímači je možné tímto způsobem dostat původní signál bez rušení, které vzniklo během přenosu 
\end{itemize}



\newpage
\section{Sběrnice -- popis}
\subsection{ISA}
\begin{itemize}
	\item asynchronní, paralelní, sdílená, nededikovaná
	\item rozšířená EISA
	\item předchůdce PCI	
\end{itemize}

\subsection{PCI}
\begin{itemize}
	\item synchronní, paralelní, sdílená, nededikovaná
	\item využívá přerušení spouštěná úrovní
	\item centralizované přidělování sběrnice
\end{itemize}

\subsection{PCI Express}
\begin{itemize}
	\item sériová, sdílená, nededikovaná	
	\item diferenciální spoj -- dvojitý jednosměrný, nízkonapěťový
	\item do systémové sběrnice byla vložena sériová komunikace
	\item \textbf{scalability:} rozšiřitelnost základního spoje s cílem dosáhnout vyššího výkonu
	\item nahrazuje AGP, PCI, PCI-X
	\item point-to-point = komunikace mezi zařízeními spojenými sběrnicí PCI-E
\end{itemize}

\subsection{SCSI}
\begin{itemize}
	\item synchronní/asynchronní, paralelní (může být i seriová), sdílená, nededikovaná
	\item distribuované přidělování sběrnice
	\item nejvyšší bit pro host adaptér (8b = 7 PZ a 1 host adapter)
	\item typ adresace: 1 z n, jedno zařízení může být obsluhováno
\end{itemize}

\subsection{SATA}
\begin{itemize}
	\item asynchronní, seriová, sdílená, nededikovaná
	\item synchronizace odvozována z dat == Embedded clock
\end{itemize}



\newpage
\section{Principy}
\subsection{Kódování 8b/10b}
\begin{itemize}
	\item 256 původně 8b znaků je zakódováno do 10 bitů
	\item 10b umožňuje 1024 kombinací -- odtud se vyberou pouze ty, které splňují požadavky
	\item Požadavek 1:
	\begin{itemize}
		\item[\textbf{?}] Co nejmenší RL(Run Length) -- dlouhé intervaly mezi změnami signálu
		\item odebrány ty kombinace, kde jsou např. samé 1 nebo samé 0	
	\end{itemize}	
	\item Požadavek 2:
	\begin{itemize}
		\item[\textbf{?}] Nulová ss složka přenášeného signálu
		\item zakódovaná data obsahují stejný počet 0 a 1
		\item nebo kapacitní člen na vstupu, který zablokuje ss složku
	\end{itemize}
	\item \emph{Disparita}
	\begin{itemize}
		\item[\textbf{?}] Rozdíl mezi počtem vyslaných 0 a 1
		\item záporná disparita = počet 0 je větší než počet 1
		\item průběžná disparita = aktuální disparita	
	\end{itemize}
	
	\item Použito v SATA a DVI, ale rozdílně
\end{itemize}


\end{document}
