\documentclass[12pt,a4paper,titlepage,final]{article}
% cestina a fonty
\usepackage[czech]{babel}
\usepackage[utf8]{inputenc}
\usepackage[top=2cm, left=1.5cm, text={18cm, 26cm}, ignorefoot]{geometry}
\usepackage{indentfirst}
\usepackage{fancyhdr}

\pagestyle{fancy}
\fancyhf{}
\lhead{ISA 2014/15}
\rhead{\textbf{Nejedná se o oficiální materiál!}}
\cfoot{\thepage}

\begin{document}
{ \huge Příprava na ISA semestrálku }

\begin{enumerate}
	\item NetFlow - vysvětlit 3 pojmy: agregace, filtrování, vzorkování
	\begin{itemize}
		\item agregace - podle klíčových položek, ukládání záznamů do Aggregation Cache
		\item filtrování - klasifikace provozu na základě hodnot v hlavičce paketů
		\item vzorkování - pro snížení nároků na hardware	
	\end{itemize}

	%%%%%%%%%%%%%%%%%%%%%%%%%%%%%%%%%%%%%%%%%%%%%%%%%%%%%%%%%%%%%%%%	
	\item Diferenciované služby - podle čeho klasifikují toky, vyjmenovat třídy
	\begin{itemize}
		\item Přednostní přeposílání -- \textbf{EF} (Expedited Forwarding)
		\item Garantované přeposílání -- \textbf{AF} (Assured Forwarding)
		\item Největší úsilí -- \textbf{BE} (Best Effort)
	\end{itemize}
	
	
	
	
	%%%%%%%%%%%%%%%%%%%%%%%%%%%%%%%%%%%%%%%%%%%%%%%%%%%%%%%%%%%%%%%%	
	\item H.323 - nakresleno schéma komunikace, popsat - doplnit příkazy, protokoly, ...
	
	%%%%%%%%%%%%%%%%%%%%%%%%%%%%%%%%%%%%%%%%%%%%%%%%%%%%%%%%%%%%%%%%	
	\item DNSSEC - řetězec důvěry, vysvětlit, jak to funguje, co je k tomu potřeba, ukázat na příkladu
	\begin{itemize}
		\item zabezpečení dat DNS pomocí asymetrické kryptografie
		\item pár veřejný-tajný klíč pro každou zónu
		\item ZSK (zone signing key) a KSK (key signing key)
		\item zabezpečení zvyšuje velikost paketu DNS
	\end{itemize}

	%%%%%%%%%%%%%%%%%%%%%%%%%%%%%%%%%%%%%%%%%%%%%%%%%%%%%%%%%%%%%%%%	
	\item 4 zkratky - vysvětlit, co znamenají, jaké použití; RTSP, MPEG TS, SDP, MCU
	\begin{itemize}
		\item Real-Time Streaming Protocol
		\begin{itemize}
			\item signalizační protokol
			\item k navázání a ukončení spojení
			\item k řízení jednoho či více časově synchronizovaných médií
			\item textový protokol, podobný HTML
		\end{itemize}

		\item MPEG Transport Stream 
		\begin{itemize}
			\item způsob kombinování (multiplexing) dat do jednoho datového toku
		\end{itemize}
		
		\item Session Description Protocol
		\begin{itemize}
			\item určený k popisu vlastností relace multimediálního přenosu dat
		\end{itemize}
		\item Multipoint Control Unit
		\begin{itemize}
			\item umožňuje pořádat vícebodové streamy	
			\item přijímá příchozí streamy a dekóduje je na příslušný výstupní stream
		\end{itemize}
	\end{itemize}
	
	%%%%%%%%%%%%%%%%%%%%%%%%%%%%%%%%%%%%%%%%%%%%%%%%%%%%%%%%%%%%%%%%	
	\item Příklad: zadána pravidla firewallu, zapsat je pomocí bitových vektorů a poté určit, které se použije na konkrétní zadanou komunikaci; jak implementovat?
	
	%%%%%%%%%%%%%%%%%%%%%%%%%%%%%%%%%%%%%%%%%%%%%%%%%%%%%%%%%%%%%%%%	
	\item Porovnejte \textbf{SNMP} a \textbf{RMON} ze všech hledisek (architektura, konfigurace atp.)
	\begin{itemize}
		\item \textbf{SNMP} -- Simple Network Management Protocol
		\begin{itemize}
			\item řídící stanice NMS, agent, databáze MIB, protokol SNMP
			\item Jazyk SMI $\rightarrow$ OID -- object ID
			\item 
		\end{itemize}	
		
		\item \textbf{RMON} -- Remote Monitoring
		\begin{itemize}
			\item sonda RMON, řídící stanice NMS, protokol SNMP	
			\item definuje speciální MIB
			\item statistiky o celém segmentu sítě
		\end{itemize}

	\end{itemize}
	%%%%%%%%%%%%%%%%%%%%%%%%%%%%%%%%%%%%%%%%%%%%%%%%%%%%%%%%%%%%%%%%
	\item Popsat princip \textbf{WRED}
	\begin{itemize}
		\item jako u RED
		\item pro každý paket ve fronte vypočte aktuální délku fronty
		\item pro nový paket dojde ke klasifikaci podle hodnoty IP precedence nebo DSCP
		\item podle třídy a aktuální délky fronty se určí pravděpodobnost zahození
	\end{itemize}

	%%%%%%%%%%%%%%%%%%%%%%%%%%%%%%%%%%%%%%%%%%%%%%%%%%%%%%%%%%%%%%%%
	\item Byl zadán nějaký konfigurák LDAPu a bylo třeba určit, co udělá. Něco společného s case sensitivitou.

	%%%%%%%%%%%%%%%%%%%%%%%%%%%%%%%%%%%%%%%%%%%%%%%%%%%%%%%%%%%%%%%%
	\item Definuj datovou strukturu trie, využití a následně příklad. Použij trie s krokem 3 a 5.

	%%%%%%%%%%%%%%%%%%%%%%%%%%%%%%%%%%%%%%%%%%%%%%%%%%%%%%%%%%%%%%%%
	\item Význam zkratek: multihoming, end-to-end, řízení zahlcení, segmentace
	\begin{itemize}
		\item Multihoming -- technika pro zvýšení spolehlivosti připojení k internetu, připojení více linkami najednou
		\item End-to-end -- ???
		\item Řízení zahlcení -- zpomaluje spojení, pokud je některá linka zahlcena.
		\item Segmentace -- ? rozdělení paketu na segmenty ?
	\end{itemize}

	%%%%%%%%%%%%%%%%%%%%%%%%%%%%%%%%%%%%%%%%%%%%%%%%%%%%%%%%%%%%%%%%
	\item Doplnit tabulku s typy záznamů DNS: SOA, RRSIG, DNSKEY, NAPTR
	\begin{itemize}
		\item \texttt{SOA} -- start of authority, obsahuje název primárního serveru a emailovou adresu správce
		\item \texttt{RRSIG} -- podpis daného záznamu
		\item \texttt{DNSKEY} -- obsahuje veřejný klíč pro ověření podpisů
		\item \texttt{NAPTR} -- mapování řetězců na data, použití REGEX na dynamické záznamy
	\end{itemize}

	%%%%%%%%%%%%%%%%%%%%%%%%%%%%%%%%%%%%%%%%%%%%%%%%%%%%%%%%%%%%%%%%
	\item Které operace jsou blokující a jak se tomu můžeme vyhnout?
	\begin{itemize}
		\item \texttt{accept()} a \texttt{recvfrom()}
		\item použití neblokujících schránek
		\item souběžné čtení více požadavků jedním procesem (\texttt{select} a \texttt{pool})
		\item řízení I/O operací pomocí signálů SIGIO 	
	\end{itemize}

	%%%%%%%%%%%%%%%%%%%%%%%%%%%%%%%%%%%%%%%%%%%%%%%%%%%%%%%%%%%%%%%%
	\item K čemu je struktura FES u simulací
	\begin{itemize}
		\item Future Event Set
		\item obsahuje množinu událostí, které se budou v simulaci zpracovávat
		\item události jsou zpracovávány na základě jejich časových razítek	
	\end{itemize}

	%%%%%%%%%%%%%%%%%%%%%%%%%%%%%%%%%%%%%%%%%%%%%%%%%%%%%%%%%%%%%%%%
	\item SIP, popsat co se děje s IP telefonem, když se připojí do sítě až k navázání komunikace a popsat protokoly
	\begin{itemize}
		\item Volající pošle \texttt{INVITE}
		\item pokud je Volaný připravený, odpovídá zprávou \texttt{OK}  	
		\item Volající potvrdí příjetí a začne komunikace
	\end{itemize}		

	%%%%%%%%%%%%%%%%%%%%%%%%%%%%%%%%%%%%%%%%%%%%%%%%%%%%%%%%%%%%%%%%
	\item Vysvětlení 5 pojmů (bitový vektor, dimenze, rozšíření prefixu, lulea) a uvést příklady použití
	\begin{itemize}
		\item \textbf{Bitový vektor} -- binárně reprezentuje určitý prefix pro lineární vyhledávání ve stromu \emph{trie}. Příklad: firewall
		\item \textbf{Dimenze} -- položky pravidla (asi)
		\item \textbf{Rozšíření prefixů} -- technika na převod prefixů určité délky na ekvivalentní prefixy větší délky. Příklad: krok 3 a 5
		\item \textbf{Lulea} -- metoda komprese vícebitového stromu \emph{trie}, odstraňuje redundantní informace vzniklé rozšiřováním prefixů.
	\end{itemize}

	%%%%%%%%%%%%%%%%%%%%%%%%%%%%%%%%%%%%%%%%%%%%%%%%%%%%%%%%%%%%%%%%
	\item Popis protokolu H.323 (225, RAS, Q.931, 245)
	\begin{itemize}
		\item zahrnuje více protokolů
		\item skládá se z: terminál, brána, ústředna, MCU
		\item \textbf{H.225} -- signalizace volání, přes ústřednu nebo přímo
		\item \textbf{RAS} -- správa, registrace, administrace volajících
		\item \textbf{Q.931} -- kanál H.225 CS pro signalizaci
		\item \textbf{H.245} -- řídící kanál 
	\end{itemize}

	%%%%%%%%%%%%%%%%%%%%%%%%%%%%%%%%%%%%%%%%%%%%%%%%%%%%%%%%%%%%%%%%
	\item Popis autentizace 802.1x s LDAP
	\begin{enumerate}
		\item Uživatel se po připojení k LAN autentizuje přes EAP
		\item Přepínač pošle autentizační požadavek serveru Radius
		\item Server Radius využije databázi LDAP pro vyhledání uživatele
		\item Po úspěšné autentizaci se klient připojí do sítě LAN
	\end{enumerate}

	%%%%%%%%%%%%%%%%%%%%%%%%%%%%%%%%%%%%%%%%%%%%%%%%%%%%%%%%%%%%%%%%
	\item Rozdělení IPv4 na třídy + privátní adresy
	\begin{itemize}
		\item třídy: A, B, C, D, E
		\item \textbf{A} -- \texttt{0-127.0.0.0 (255.0.0.0)}
		\item \textbf{B} -- \texttt{128-191.0.0.0 (255.255.0.0)}
		\item \textbf{C} -- \texttt{192-223.0.0.0 (255.255.255.0)}
		\item \textbf{D} -- \texttt{224-239.0.0.0 (255.255.255.255)}
		\item \textbf{E} -- \texttt{240-255.0.0.0 (...)}   	
	\end{itemize}

	%%%%%%%%%%%%%%%%%%%%%%%%%%%%%%%%%%%%%%%%%%%%%%%%%%%%%%%%%%%%%%%%
	\item Multicast pomocí socketů (jak, příkazy, atd..)
	\begin{itemize}
		\item přenáší IP datagramy na skupinu počítačů s jedinou IP adresou
		\item cílová adresa třídy D
		\item posílá UDP pakety
		\item podporované schránky: \texttt{AF\_INET}, \texttt{SOCK\_DGRAM}, \texttt{SOCK\_RAW}
		\item implicitně TTL = 1
		\item nastavení multicastu pomocí \texttt{setsockopt()}  
	\end{itemize}		

	%%%%%%%%%%%%%%%%%%%%%%%%%%%%%%%%%%%%%%%%%%%%%%%%%%%%%%%%%%%%%%%%
	\item Adresování a směrování VoIP, VoD - co to je, příkazy..
	\begin{itemize}
		\item \textbf{SIP -- Session Initiation Protocol}
		\begin{itemize}
			\item adresování prováděno pomocí SIP URI: \texttt{sip:user@domain}
			\item směrovací informace uloženy v SIP hlavičce paketu
			\item směrování je prováděno SIP servery po cestě
		\end{itemize}
		\item \textbf{VoD -- Video on Demand} 
		\begin{itemize}	
			\item klientovi je zasílán předem vytvořený multimediální obsah
			\item přehrávání lze řídit
		\end{itemize}
	\end{itemize}

	%%%%%%%%%%%%%%%%%%%%%%%%%%%%%%%%%%%%%%%%%%%%%%%%%%%%%%%%%%%%%%%%
	\item 2-dimenzionální trie
	\begin{itemize}
		\item pro každou dimenzi strom trie
		\item exponenciální nárůst prostorové složitosti
		\item ???	
	\end{itemize}

	%%%%%%%%%%%%%%%%%%%%%%%%%%%%%%%%%%%%%%%%%%%%%%%%%%%%%%%%%%%%%%%%
	\item Teorie (OSI, subnetting, PDU, SLAAC - vše stručně popsat)
	\begin{itemize}
		\item \textbf{OSI} -- 7 vrstev, každá vrstva definuje funkci protokolu, atp.
		\item \textbf{Subnetting} -- rozdělování sítě na podsítě, pomocí prefixu
		\item \textbf{PDU} -- datová jednotka modelu OSI
		\item \textbf{SLAAC} -- bezstavová autokonfigurace adres
	\end{itemize}
	
	%%%%%%%%%%%%%%%%%%%%%%%%%%%%%%%%%%%%%%%%%%%%%%%%%%%%%%%%%%%%%%%%
	\item Záznam v LDAP, co obsahuje, čím je identifikován, uvést příklad
	\begin{itemize}
		\item alternativa k X.500
		\item stejný koncept uspořádání dat jako X.500
		\item jednodušší implementace, jeden přenosový protokol
	    \item skládá se ze záznamů (entry):
	    \begin{itemize}
	        \item LDAP je jednoznačně identifikován identifikátorem \textbf{DN} 	
		    \item obsahuje seznam atributů ve tvaru (typ, hodnota)
		\end{itemize}
		\item atributy:
		\begin{itemize}
			\item popisuje vlastnost objektu ve tvaru (typ, hodnota)
			\item definuje jak se hodnoty porovnávají
			\item hodnota atributu musí splňovat syntax danou typem	
		\end{itemize}
	\end{itemize}

	%%%%%%%%%%%%%%%%%%%%%%%%%%%%%%%%%%%%%%%%%%%%%%%%%%%%%%%%%%%%%%%%
	\item Čtyři problémy u IP telefonie (jitter, zpoždění, echo, ztrátovost) v tabulce, napsat popis o co jde a jak to snížit
	\begin{figure}[h!]
	\centering
	\begin{tabular}{| l | p{5cm} | p{5cm} |}
		\hline
		\textbf{Problém} & \textbf{Popis} & \textbf{Řešení} \\ \hline \hline
		Jitter & Rozptyl paketů & použití vyrovnávacích bufferů \\ \hline
		Zpoždění & ... & musí být zajištěna prioritizace hlasových paketů \\ \hline
		Echo & Ozvěna & eliminována aktivním potlačením DSP (echo suppression) \\ \hline
		Ztrátovost & destruktivní účinky u kodeků s prediktivními metodami & \\ \hline
	\end{tabular}
	\end{figure}

	%%%%%%%%%%%%%%%%%%%%%%%%%%%%%%%%%%%%%%%%%%%%%%%%%%%%%%%%%%%%%%%%
	\item Pojmy z teorie - FCAPS, MIB, SMI, + Pasivní monitoring sítě
	\begin{itemize}
		\item \textbf{FCAPS} - model nad ISO Telecommunications Management Network
		\begin{itemize}
			\item \textbf{F} -- Fault -- detekce, izolace a oprava chyb, testy
			\item \textbf{C} - Configuration -- sledování připojených zařízení, DB kofigurací monitorovaných objektů, kontroly stavu
			\item \textbf{A} - Accounting -- správa uživatelských účtů a poplatků, logování, řízení přístupu, kvóty, statistiky a identifikace přenosů
			\item \textbf{P} - Performance -- monitorování objektů, řízení výkonosti, plánování zdrojů
			\item \textbf{S} - Security -- definice bezpečnostní politiky, řízení přístupu ke zdrojům, klíče, certifikáty
		\end{itemize} 	
		\item \textbf{MIB} -- Management Information Base, používáno jako zdroj dat pro NMS
		\item \textbf{SMI} - Structure Management Information -- hazyk pro definici pravidel na vytváření objektů SNMP popsané notací ASN.1.
		
		\item Pasivní monitoring sítě:
		\begin{itemize}
			\item využívá především logovací informace aplikací a služeb sbírané u administrátora
			\item např. \texttt{syslog} 	
		\end{itemize}
	\end{itemize}

	%%%%%%%%%%%%%%%%%%%%%%%%%%%%%%%%%%%%%%%%%%%%%%%%%%%%%%%%%%%%%%%%
	\item SNMP jaký věci se dají zjišťovat pro tyhle skupiny objektů - system, interface, ip, tcp. Pro každý aspoň tři příklady
	\begin{itemize}
		\item \textbf{System:} jméno OS, systémový čas, kontakt
		\item \textbf{Interface:} stav síťového rozhraní
		\item \textbf{IP:} IP adresa, směrovací informace
		\item \textbf{TCP:} stav spojení TCP (closed, listen, synSent) 
	\end{itemize}

	%%%%%%%%%%%%%%%%%%%%%%%%%%%%%%%%%%%%%%%%%%%%%%%%%%%%%%%%%%%%%%%%
	\item Popsat základní části videokonferenčího terminálu	
	\begin{itemize}
		\item koncové zařízení
		\item zajišťuje obousměrnou komunikaci s jiným koncovým zařízením
		\item složky:
		\begin{itemize}
			\item \textbf{Videokamera} -- snímání scény, PAL, NTSC, HDV
			\item \textbf{Zobrazovací plocha} -- monitor, LCD, projektor
			\item \textbf{Mikrofon} -- zvuk je důležitější než obraz, vhodný mix.pult
			\item \textbf{Reproduktory} -- součástí zobrazovacího zařízení, sluchátka, vlastní audio systém
			\item \textbf{Kodek} -- SW/HW, komunikace s ostatními zařízeními
			\item \textbf{UI} -- řízení konference na straně uživatele, dálkový ovladač+menu, webové rozhraní	
		\end{itemize}	
	\end{itemize}

\end{enumerate}

\newpage
{\LARGE Příklady: pokusy o výpočet}
\begin{enumerate}
	%%%%%%%%%%%%%%%%%%%%%%%%%%%%%%%%%%%%%%%%%%%%%%%%%%%%%%%%%%%%%%%%	
	\item Příklad RED - zadáno Qmin a Qmax, výpočet pravděpodobnosti, že dorazivší paket bude zahozen.
	\[ 
		P_a = P_{max} * \frac{Q_{avg} - Q_{min}}{Q_{max} - Q_{min}}
	\]
	
	
	%%%%%%%%%%%%%%%%%%%%%%%%%%%%%%%%%%%%%%%%%%%%%%%%%%%%%%%%%%%%%%%%
	\item Spočítat potřebnou kapacitu pro NetFlow záznam na lince za 1 hodinu. \\
	\emph{Linka má rychlost 4Gbps, velikost jednoho tou je 20000B, spočítat NetFlow za hodinu}
	
	$$ 4Gb = 512 MB = 512 000 000 B  $$
	$$ \frac{512 000 000}{20 000} = 25 000 toku $$	
	$$ 25 000 * 3600s * 50B = 4.5 GB za hodinu $$ 
	


	%%%%%%%%%%%%%%%%%%%%%%%%%%%%%%%%%%%%%%%%%%%%%%%%%%%%%%%%%%%%%%%%
	\item Frekvence 4kHz před vzorkováním, na 4b vzorek, režie 58B, posílá se každých 20ms, jaké potřebujeme přenosové pásmo?
	
	Vzorkovací teorém: $4kHz -> 8kHz$ \\
	Šířka pásma kodeku: $8k * 4b = 32kbps$ \\
	$32kb * 0.02 = 640b = 80B$ \\
	$80B + 58B = 138B$, $f = \frac{1}{T} = \frac{1}{0.02} = 50 pps$ \\
	$50*138 = 6900 B/s = 55.2 kbps$
	

	%%%%%%%%%%%%%%%%%%%%%%%%%%%%%%%%%%%%%%%%%%%%%%%%%%%%%%%%%%%%%%%%
	\item Příklad na výpočet přenosovýho pásma + jaké protokoly obsahuje odchozí paket
	\begin{itemize}
		\item RTP, UDP, IP, Ethernet	
	\end{itemize}


	%%%%%%%%%%%%%%%%%%%%%%%%%%%%%%%%%%%%%%%%%%%%%%%%%%%%%%%%%%%%%%%%
	\item Příklad na DNS, mám firmu s mailovým serverem, name serverem, WWW serverem a 10 stanicema. Mám nějakej rozsah IP adres a napsat všecky záznamy(i pro reverzní mapování) počtače n1-n10, adresní prostor 81.94.23.64/28, doména www.nejakadomena.cz, DNS server ns.nejakadomena.cz, Mail mail.nejakadomena.cz

	\item Bitový vektor: Implementujte níže uvedená pravidla firewallu (a) pomocí bitového vektoru, (b) pomocí kartézského součinu rekurzivní klasifikace RFC s třídami ekvivalence (srcIP-dstIP) a (SrcPort-DstPort).
	\begin{verbatim}
	R1: permit TCP from 147.229.0.0 to any dst-port 80
	R2: permit UDP from 147.229.0.0 to any dst-port 53
	R3: permit UDP from any to 147.229.0.0 src-port 53
	R4: permit ICMP from 147.229.0.0 to any
	R5: deny ICMP from 147.229.1.15 to any
	R6: deny IP from any to any
	\end{verbatim}
\begin{table}[h!]
	\centering	
	\begin{tabular}{| c | c || c | c || c | c || c | c |}
		\hline
		\textbf{SRC} & \textbf{BitV} & \textbf{DST} & \textbf{BitV} & \textbf{SRC Port} & \textbf{BitV} & \textbf{DST Port} & \textbf{BitV} \\ \hline \hline
		147.229.0.0 & 111101 & 147.229.0.0 & 111101 & 53 & 111111 & 80 & 101111 \\ \hline
		149.229.1.15 & 000011 & 147.229.1.15 & 110011 & * & 110111 & 53 & 011111 \\ \hline
		* & 001001 & * & 110001 & & & * & 001111 \\ \hline 	
	\end{tabular}
	
	\begin{tabular}{| c | c |}
		\hline
		\textbf{Protocol} & \textbf{BitV} \\ \hline \hline
		TCP & 100001 \\ \hline
		UDP & 011001 \\ \hline
		ICMP & 000111 \\ \hline
		IP & 000001 \\ \hline
	\end{tabular}
	\caption{Řešení a - bitové vektory}



	\begin{tabular}{| c | c | c |}
		\hline
		\textbf{SrcIP - DstIP} & \textbf{bitV} & \textbf{class} \\ \hline \hline
		(147.229.0.0, 147.229.0.0) & 111101 & C1 \\ \hline
		(147.229.0.0, 147.229.1.15) & 110001 & C2 \\ \hline
		(147.229.0.0, *) & 110001 & C2 \\ \hline
		(147.229.1.15, 147.229.0.0) & 000001 & C3 \\ \hline
		(147.229.1.15, 147.229.1.15) & 000011 & C4 \\ \hline
		(147.229.1.15, *) & 000001 & C3 \\ \hline
		(*, 147.229.0.0) & 000001 & C3 \\ \hline
		(*, 147.229.1.15) & 000001 & C3 \\ \hline
		(*, *) & 000001 & C3 \\ \hline \hline
		\textbf{SrcPort - DestPort} & \textbf{bitV} & \textbf{class} \\ \hline \hline
		(53, 80) & 101111 & D1 \\ \hline
		(53, 53) & 011111 & D2 \\ \hline
		(53, *) & 001111 & D3 \\ \hline
		(*, 80) & 100111 & D4 \\ \hline
		(*, 53) & 010111 & D5 \\ \hline
		(*, *) & 000111 & D6 \\ \hline
	\end{tabular}
	\caption{Řešení b - karétzský součin}
\end{table}




\end{enumerate}
\end{document}
